\documentclass[]{article}
\usepackage{lmodern}
\usepackage{amssymb,amsmath}
\usepackage{ifxetex,ifluatex}
\usepackage{fixltx2e} % provides \textsubscript
\ifnum 0\ifxetex 1\fi\ifluatex 1\fi=0 % if pdftex
  \usepackage[T1]{fontenc}
  \usepackage[utf8]{inputenc}
\else % if luatex or xelatex
  \ifxetex
    \usepackage{mathspec}
  \else
    \usepackage{fontspec}
  \fi
  \defaultfontfeatures{Ligatures=TeX,Scale=MatchLowercase}
\fi
% use upquote if available, for straight quotes in verbatim environments
\IfFileExists{upquote.sty}{\usepackage{upquote}}{}
% use microtype if available
\IfFileExists{microtype.sty}{%
\usepackage{microtype}
\UseMicrotypeSet[protrusion]{basicmath} % disable protrusion for tt fonts
}{}
\usepackage[margin=1in]{geometry}
\usepackage{hyperref}
\hypersetup{unicode=true,
            pdftitle={Walmart Interview Answers},
            pdfborder={0 0 0},
            breaklinks=true}
\urlstyle{same}  % don't use monospace font for urls
\usepackage{graphicx,grffile}
\makeatletter
\def\maxwidth{\ifdim\Gin@nat@width>\linewidth\linewidth\else\Gin@nat@width\fi}
\def\maxheight{\ifdim\Gin@nat@height>\textheight\textheight\else\Gin@nat@height\fi}
\makeatother
% Scale images if necessary, so that they will not overflow the page
% margins by default, and it is still possible to overwrite the defaults
% using explicit options in \includegraphics[width, height, ...]{}
\setkeys{Gin}{width=\maxwidth,height=\maxheight,keepaspectratio}
\IfFileExists{parskip.sty}{%
\usepackage{parskip}
}{% else
\setlength{\parindent}{0pt}
\setlength{\parskip}{6pt plus 2pt minus 1pt}
}
\setlength{\emergencystretch}{3em}  % prevent overfull lines
\providecommand{\tightlist}{%
  \setlength{\itemsep}{0pt}\setlength{\parskip}{0pt}}
\setcounter{secnumdepth}{0}
% Redefines (sub)paragraphs to behave more like sections
\ifx\paragraph\undefined\else
\let\oldparagraph\paragraph
\renewcommand{\paragraph}[1]{\oldparagraph{#1}\mbox{}}
\fi
\ifx\subparagraph\undefined\else
\let\oldsubparagraph\subparagraph
\renewcommand{\subparagraph}[1]{\oldsubparagraph{#1}\mbox{}}
\fi

%%% Use protect on footnotes to avoid problems with footnotes in titles
\let\rmarkdownfootnote\footnote%
\def\footnote{\protect\rmarkdownfootnote}

%%% Change title format to be more compact
\usepackage{titling}

% Create subtitle command for use in maketitle
\providecommand{\subtitle}[1]{
  \posttitle{
    \begin{center}\large#1\end{center}
    }
}

\setlength{\droptitle}{-2em}

  \title{Walmart Interview Answers}
    \pretitle{\vspace{\droptitle}\centering\huge}
  \posttitle{\par}
    \author{}
    \preauthor{}\postauthor{}
    \date{}
    \predate{}\postdate{}
  

\begin{document}
\maketitle

\hypertarget{r-markdown}{%
\subsection{R Markdown}\label{r-markdown}}

This is an R Markdown document. Markdown is a simple formatting syntax
for authoring HTML, PDF, and MS Word documents. For more details on
using R Markdown see \url{http://rmarkdown.rstudio.com}.

When you click the \textbf{Knit} button a document will be generated
that includes both content as well as the output of any embedded R code
chunks within the document. You can embed an R code chunk like this:

library(data.table) \#\# for importing data library(tidyverse) \#\# for
data wrangling library(stringr) \#\# for string manipulation
library(plotly) library(dplyr) library(Hmisc)

\#\#set working directory setwd(`C:/OneDrive/OneDrive -
Tredence/Attachments/Omni Channel-Walmart
Interview/Interview/ml-latest-small')

links\textless{}-read.csv(`links.csv',stringsAsFactors = F)
movies\textless{}-read.csv(`movies.csv',stringsAsFactors = F)
ratings\textless{}-read.csv(`ratings.csv',stringsAsFactors = F)
tags\textless{}-read.csv(`tags.csv',stringsAsFactors = F)

\#\#Data Manipulations before solving the questions \#\#Exrtact year and
title out of title column and split genre in movies df
movies\textless{}-movies \%\textgreater{}\%\\
mutate(year=as.numeric(substr(title,unlist(regexpr(``\textbackslash{}({[}0-9\textbackslash{}-{]}\emph{\textbackslash{})\(",title))+1,unlist(regexpr("\\([0-9\\-]*\\)\)",title))+4)),
title\_new=substr(title,1,ifelse(unlist(regexpr("\textbackslash{}({[}0-9\textbackslash{}-{]}}\textbackslash{})\(",title))-1==-2,nchar(title),unlist(regexpr("\\([0-9\\-]*\\)\)'',title))-1)),
genre=strsplit(genres,``\textbackslash{}\textbar{}''))
\%\textgreater{}\% unnest(genre)

movies\textless{}-movies{[}complete.cases(movies),{]}

\#\#convert unix timestamp to standard dates
tags\textless{}-tags\%\textgreater{}\%
mutate(tags\_Date=as.Date(as.POSIXct(timestamp,origin=``1970-01-01'',tz=``GMT'')),
tags\_year=year(tags\_Date))

ratings\textless{}-ratings\%\textgreater{}\%
mutate(rat\_Date=as.Date(as.POSIXct(timestamp,origin=``1970-01-01'',tz=``GMT'')),
rat\_year=year(rat\_Date))

\hypertarget{the-hypothesis}{%
\subsection{The Hypothesis:}\label{the-hypothesis}}

\#\#" count of movie produced per year depends on rating and total
reviews of the movies of past years by genre- \#\#We will be doing this
for most popular genre and from 2000-2015. Popular would be decided by
the maximum number \#\#of movies across years"

\hypertarget{steps-to-be-followed}{%
\subsection{Steps to be followed:}\label{steps-to-be-followed}}

\hypertarget{find-the-most-popular-genre}{%
\section{1. Find the most popular
genre}\label{find-the-most-popular-genre}}

\hypertarget{find-total-number-of-movies-made-in-that-genre-in-2000-2015}{%
\section{2. Find total number of movies made in that genre in
2000-2015}\label{find-total-number-of-movies-made-in-that-genre-in-2000-2015}}

\hypertarget{find-avg-rating}{%
\section{3. Find avg rating}\label{find-avg-rating}}

\hypertarget{cross-reference-step-2-and-3-to-find-out-the-nature-of-the-relation-if-any}{%
\section{4. Cross reference step 2 and 3 to find out the nature of the
relation, if
any}\label{cross-reference-step-2-and-3-to-find-out-the-nature-of-the-relation-if-any}}

\#\#Assumptions: \#\#Data is holistic to be used for the analysis

\#1.Find the most popular genre

genre\textless{}-movies\%\textgreater{}\%
distinct(movieId,genre)\%\textgreater{}\%
group\_by(genre)\%\textgreater{}\%
summarise(count=n())\%\textgreater{}\%
filter(row\_number(desc(count))==1) \#\# genre is drama

\hypertarget{find-total-number-of-movies-made-in-that-genre-in-2000-2015-1}{%
\section{2. Find total number of movies made in that genre in
2000-2015}\label{find-total-number-of-movies-made-in-that-genre-in-2000-2015-1}}

movie\_per\_year\textless{}-movies\%\textgreater{}\% distinct(
movieId,year,genre)\%\textgreater{}\% filter(year \textgreater{}= 2000
\& year \textless{}= 2015)\%\textgreater{}\% filter(genre
==`Drama')\%\textgreater{}\% group\_by(year)\%\textgreater{}\%
summarise(movie\_count=n())

\hypertarget{find-avg-rating-1}{%
\section{3. Find avg rating}\label{find-avg-rating-1}}

rat\_per\_year\textless{}- movies\%\textgreater{}\% distinct(
movieId,year,genre)\%\textgreater{}\% filter( genre ==
``Drama'')\%\textgreater{}\% filter(year \textgreater{}= 2000 \& year
\textless{}= 2015)\%\textgreater{}\% left\_join(ratings, by =
``movieId'')\%\textgreater{}\% group\_by(year)\%\textgreater{}\%
summarise(avg\_rating=mean(rating,na.rm = T),
rev\_count=n\_distinct(userId))

\hypertarget{joining-the-2}{%
\subsection{Joining the 2}\label{joining-the-2}}

count\_rat\_per\_year\textless{}-left\_join(movie\_per\_year,rat\_per\_year)

plot1\textless{}-count\_rat\_per\_year\%\textgreater{}\% ggplot()+
geom\_line(aes(x=year,y=movie\_count,color=``red''))+
geom\_line(aes(x=year,y=rev\_count,color=``blue'' ))+
geom\_line(aes(x=year,y=avg\_rating,color=``green'' ))

ggplotly(plot1)

rcorr(as.matrix(count\_rat\_per\_year)) \#\#\# From data and chart its
clear that there is more or less no impact of \#\# avg ratings and
number of reviews on the count of movies being made

\begin{verbatim}

## Including Plots

You can also embed plots, for example:

plot1<-count_rat_per_year%>%
      ggplot()+ geom_line(aes(x=year,y=movie_count,color="red"))+
      geom_line(aes(x=year,y=rev_count,color="blue" ))+
      geom_line(aes(x=year,y=avg_rating,color="green" ))

ggplotly(plot1)
\end{verbatim}

Note that the \texttt{echo\ =\ FALSE} parameter was added to the code
chunk to prevent printing of the R code that generated the plot.


\end{document}
